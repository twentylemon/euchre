
\section{Introduction}
% talk about related work here

A good heuristic or static evaluation function in games helps an agent make decisions which maximize their utility. This idea
works well, however in team settings, maximizing a personal score may not be a good team strategy. In this project, we
investigate this setting in the card game euchre, where two teams of two complete to maximize their team score. Some agents
show, among those studied, that cooperation is a much more powerful strategy. This idea was defined in the prisoner's dilemma, where
two rational agents that maximize their personal score end up both losing. If these agents cooperated instead, their personal scores
and ``team'' scores would be optimal.

Card games remain popular to study, as their offer interesting settings which lead to good understandings and findings. For example,
a common thread in many card games is that they are incomplete information games -- other players' hands or the deck is not known to
the agent. Perhaps the most interesting about card games is that humans are (generally) much better than the best known artificial agent.
Famously, poker was solved in \cite{poker}, however they study a very small version of poker compared to what is normal in tournament play.
Not to discredit their results, but just to say that much work needs still be done in order for artificial agents to consistently beat the
best human players in many card game settings. Euchre, compared to other card games, is less conventional in that it is also a cooperative game.
This project details various artificial agents for euchre in order to discover a strategy that is superior.

\subsection{Euchre}

Simply put, euchre is a trick taking card game. It plays very similarly to other trick taking games, such as hearts or bridge.
The euchre deck only consists of 24 cards, 9 through ace of each suit. Each of the four players is dealt five cards which makes up
their hand. Each players' hand is only known to them. The 21st card of the deck is turned face up to be offered as trump. The remaining
three cards remain hidden and are not used in the hand (known as the ``kitty'').

Players then take turns deciding on if they wish to ``order up'' the offered trump card. If so, the dealer must pick up the offered card,
place it in his hand and secretly discard a card. If the card is not ordered up, players take turns having the option to call a trump suit
or pass -- but players cannot call the same trump suit as what was offered prior. Players can also choose to ``go alone'' when they call
a trump suit, which forbids their partner from playing the hand in the hopes of scoring more points.

Once trump is called, the main portion of the game takes place. The player left of the dealer leads the first trick. Each player adds a card
to the trick in turn. The winner of the trick leads the next trick. This play continues for five rounds, until all players are out of cards.
Afterwards, teams count how many trick their won, and the team that took at least 3 of the tricks scores points as:
\begin{itemize}[noitemsep, label={}]
    \item 1 point : if called trump and took 3-4 tricks
    \item 2 points: if called trump and took all 5 tricks
    \item 2 points: if didn't call trump and took 3-5 tricks
    \item 4 points: if went alone and took all 5 tricks
\end{itemize}

This play continues, shifting who deals every hand, until a team scores 10 points. That is euchre in a nutshell -- though trump hasn't been explained.
The general play follows the above guidelines. There are additional rules with trump and which cards you can play into a trick, which are explained below.
\\
When a player leads a trick, the card they play determines the lead suit of the trick. Other players must play a card of the same suit
if possible, otherwise they can play any card from their hand. The player with the highest ranked card in the trick wins it, however those cards
which are not the lead suit or trump cannot ever win the trick. For example, if I lead the $9\heartsuit$ and you followed with A$\diamondsuit$, my
lowly 9 is still winning the trick because you did not follow suit (assuming $\diamondsuit$ is not trump).
\\
Trump in euchre is peculiar, and leads to much confusion for beginners. For non-trump suits, the rankings of cards is as you would expect, ace $>$
king $>$ queen \ldots. The lowest trump cards beat the highest non-trump cards (easy enough), however the rankings in among trump is not the same as
non-trump. Let $\spadesuit$ be trump for a concrete example. The ordering of trump then goes
J$\spadesuit$ $>$ J$\clubsuit$ $>$ A$\spadesuit$ $>$ K$\spadesuit$ $>$ Q$\spadesuit$ $>$ $10\spadesuit$ $>$ $9\spadesuit$.
The jack is the highest card in the game, followed by the other jack of the same colour -- in our example J$\spadesuit$ is the best and J$\clubsuit$
is the second best (called the ``right'' and ``left'' bowers respectfully). The left bower is the confusing part. Though it is printed J$\clubsuit$, it
becomes a $\spadesuit$ as long as $\spadesuit$ is trump. So when $\clubsuit$ is led and your only $\clubsuit$ is J$\clubsuit$, you are not forced
to play it as it is not a $\clubsuit$, but a $\spadesuit$. Likewise, when $\spadesuit$ is led, you must play J$\clubsuit$ since it is a $\spadesuit$.
This means there are 7 $\spadesuit$ in the game, 6 of each red suit and only 5 $\clubsuit$.
Similarly when a red suit is called trump, the other red jack becomes the left bower and it becomes on the trump suit rather than it's printed suit.


\subsection{Agent Considerations}

With the rules of euchre in mind, we can consider what information is known and important to an artificial agent. When it comes times to play a card
into a trick, the following information is known:
\begin{itemize}[noitemsep, label={}]
    \item the rules of euchre
    \item their own hand
    \item the card offered as trump
    \item all of the cards played so far, and by whom
    \item the trump suit
    \item the lead suit of the trick, is any
    \item the overall rankings of each card
\end{itemize}

This is all the information an agent has access to. Additional information can be inferred, such as when a player doesn't follow the lead suit,
it is known that the player cannot possibly have any cards of that suit in their hand. Making the best use of all the information available
will lead to a powerful euchre agent. In the next section, we introduce several euchre agents which make different uses of all this available
information. Afterwards, we compare these strategies to each other to determine which is best among them. Lastly, we will discuss the
strengths and weaknesses of particular strategies and conclude with ideas for future work.
