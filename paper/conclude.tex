
\section{Conclusions and Future Work}

This project aimed to create an intelligent euchre playing agent, but found that relatively simple strategies performed
just as well if not better than more complex ones. This leads to the conclusion that euchre is mostly not about trick taking,
but about trump calling. Unfortunately, none of the strategies given are able to call a meaningful trump suit. This is not to say
that CoopHighLow, the best performing agent overall, is an optimal trick taking euchre strategy. On the contrary, we have given some
weaknesses that the CoopHighLow strategy has and ways to improve upon them.

Some future work would be to create a more complete rule set for the CardCoutning agent to improve it. This would in turn improve
the Hybrid agent. Further improvements can be made to Hybrid, the parameters can be learned instead of arbitrarily assigned, perhaps
leading to a much better agent overall. Perhaps the learned values can be used to make an intelligent decision when calling trump.
Additionally, a more complex Markov decision process can be used instead of a simple threshold.

In this project, several trick taking euchre strategies are given. A simple rule base consisting of Random, High and Low build
up into more complex and more powerful agents. Counter intuitively, a relatively simple strategy proves to be the best trick taker
out of those studied. Though it is not a perfect strategy, it can serve as an excellent foundation for future trick taking strategies.
This result leads to the conclusion that trick taking is euchre is not as difficult as the decision to call trump, when very little
about the state of the game is known.

All of the source code for this project is freely available at\\https://github.com/twentylemon/euchre
