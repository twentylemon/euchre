
\section{Euchre Agents}

The overall objective in euchre is to take as many tricks as possible. This ensures the other team does not benefit from them.
Generally, this strategy works well, but it ignores the team aspect of the game. In this section, we give several euchre strategies
making various use of the information available in the game. All agents know the basic rules of the game, as a useful agent could
not be constructed without this knowledge. Additionally, no agent can cheat. When we say an agent chooses to play a card with some
property, it is assumed that the card is chosen from the set of legally playable cards. These strategies range from simple to selfish to
complex. These strategies aimed to produce a very quick but good decision when playing a card from their hand into a trick.


\subsection{Simple Strategies}

The first strategies discussed will be very simple, but serve as a very good base for more complex strategies. These agents only make use
of the information given directly to them, their hand. The first agent plays a random card from their hand. This agent exists to serve
as a basis for comparison. The next two agents are opposites of each other, one always plays the lowest card in their hand while
the other always plays the highest card. Since these agents do not use any additional information, they all play selfishly.

The agents High and Low serve as an excellent foundation for more complex agents. In euchre, the overall goal is to maximize the number
of tricks your team earns. Playing High will give you the best possible chance of winning a trick, while Low allows you to throw away
your worst card if you resign a trick. Playing a ``Middle'' strategy would not help compared to High and Low as it seems to give you
the worst traits of both the other strategies. With Middle, you could have an increased chance of taking a trick, but if you lost the trick,
you've lost a decent card which could have been used later to win a different trick. Do to this, no Middle agent was explored.


\subsection{More Complex Strategies}

The next agents become slightly more complex. These agents make use of the information given to them in the trick. The first one,
called HighLow, behaves as follows:
\begin{itemize}[noitemsep, label={}]
    \item if I have a card that can win the trick, play High
    \item otherwise, there's no hope, play Low
\end{itemize}
This HighLow strategy proves to perform quite well, and makes a lot of sense. If it is possible to take a trick, playing the best card
gives the highest chance of actually winning the trick. If there's no hope of winning a trick, playing your worst card
saves your better cards in the hopes that they can win later tricks.

A cooperative version of HighLow was created as well, called CoopHighLow. The logic of this agent is as follows:
\begin{itemize}[noitemsep, label={}]
    \item if partner has played in the trick and is winning, play Low
    \item otherwise, play HighLow
\end{itemize}
This strategy aims to not take tricks from your partner if it can be avoided. If your partner is winning the trick, saving your good cards for
later and allowing them to win will generally be a very powerful strategy. However, if your partner is losing the trick, or hasn't played,
the semi-aggressive HighLow strategy makes sense to play to try to win tricks or avoid losing better cards.


\subsection{Markov Decision Strategy}



\subsection{Card Counting Strategy}



\subsection{Hybrid Strategy}


\subsection{Monte Carlo Agent}

