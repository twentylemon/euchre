
\section{Discussion}

Euchre is a game where the goal is to maximize the number of tricks your team takes. With this in mind, it is not
totally surprising that such a simple strategy such as CoopHighLow performs so well. This strategy aims, in a very simple way,
to maximize the number of tricks the team takes. When it is possible to win a trick, try as much as possible to win, otherwise
playing the lowest card saves your better cards in hopes they can win later tricks. The CardCounting and Hybrid strategies also
perform very well in this regard, but do not prove to be any better than the simplistic CoopHighLow. This section discusses
strengths and weaknesses of particular agents, and what could be done better in the future.


\subsection{Strengths of Particular Agents}

First of all, praise needs to be given to an underrated agent: Low. Low by itself performed worse than every other agent, however
this agent is a key ingredient when creating powerful agents. The Low strategy can be summed up as ``I'll save this for later!''
This isn't necessarily a strength of the Low agent, but it's value should not be taken for granted.

It's the passive value of Low combined with the aggressive value of High, along with some team play, that makes CoopHighLow such
a good strategy. As mentioned, CoopHighLow strives to win tricks were it can as long as it's not hindering the team. This idea
can prove a great foundation for any euchre playing agent.

The CardCounting agent shows more promise compared to other agents -- it is a strategy that actually uses all available information
to it's advantage. The particular rules followed could be improved upon to create a more useful and intelligent agent.

MonteCarlo obviously has the most potential for perfect play. In fact, nearing the end of the game, MonteCarlo10 should be close to perfect play,
if not perfect already. This is because as the game is played, more and more information becomes known. As mentioned, there are about 3billion
branches in the game tree at the beginning of the game. When leading a trick with only three cards left, the tree size is reduced drastically to
$3 {12 \choose 3}{9 \choose 3}{6 \choose 3} \approx 10^6$. This tree size becomes easily traversable, all games can be played out before
making a decision if given a reasonable amount of time.


\subsection{Weaknesses of Particular Agents}

For the praise of Low, there's also an inherent flaw. When playing Low, you save good cards for later but you are unlikely to lead a trick.
If you do not lead a trick, high non-trump suit cards can potentially be worthless. This is why Low by itself performs so poorly. Pairing
Low with High removes the flaws of Low.

CoopHighLow, even being the best agent overall, plays very poorly sometimes. Particularly, it is very when leading a trick. Blindly following
the rules, CoopHighLow will lead with it's highest card. This means leading trump frequently, which most human players will tell you is a bad play.
For example, CoopHighLow will happily lead it's $9\spadesuit$ as it's highest card, but this is the lowest trump -- this type of play is essentially
asking other plays to freely take his trump away.

It is for this reason CardCounting as a high potential. It can avoid those type of mistakes using it's more complex rule set. The proposed rule set
seems to not be perfect however. Additional rules may need to be added or some rules may require changes. Perhaps the largest flaw with the
CardCounting agent is that it fails to ever check if opponents are missing a suit. For example, when leading CardCounting checks if their partner
is out of a suit and tries to lead that suit hoping the partner can trump it -- but perhaps this rule should not be followed if the opponents are also
out of the suit and are stronger in trump than my partner. A more complex rule set may lead to a more well rounded agent. Additionally, perhaps
a more rigorous definition of ``strong'' would be required for this agent to perform better.

The Markov agents, and by extension the Hybrid agents suffer from one major flaw: they are parametrized. The values of each card and for $\tau$
need to be assigned. In our experiments, somewhat natural values were assigned, however a natural value may not lead to the best performing agent.

The MonteCarlo agents, as discussed previously, cannot search enough of the game tree in the early game. This may lead to poor decisions early
which snowball into an unwinnable late game.

